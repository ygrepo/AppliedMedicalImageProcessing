\documentclass[12pt,twoside]{article}
\usepackage[dvipsnames]{xcolor}
\usepackage{tikz,graphicx,amsmath,amsfonts,amscd,amssymb,mathrsfs, bm,cite,epsfig,epsf,url}
\usepackage[hang,flushmargin]{footmisc}
\usepackage[colorlinks=true,urlcolor=blue,citecolor=blue]{hyperref}
\usepackage{amsthm,multirow,wasysym,appendix}
\usepackage{array,subcaption} 
% \usepackage[small,bf]{caption}
\usepackage{bbm}
\usepackage{pgfplots}
\usetikzlibrary{spy}
\usepgfplotslibrary{external}
\usepgfplotslibrary{fillbetween}
\usetikzlibrary{arrows,automata}
\usepackage{thmtools}
\usepackage{blkarray} 
\usepackage{textcomp}
\usepackage{float}
%\usepackage[left=0.8in,right=1.0in,top=1.0in,bottom=1.0in]{geometry}


\usepackage{times}
\usepackage{amsfonts}
\usepackage{amsmath}
\usepackage{latexsym}
\usepackage{color}
\usepackage{graphics}
\usepackage{enumerate}
\usepackage{amstext}
\usepackage{blkarray}
\usepackage{url}
\usepackage{epsfig}
\usepackage{bm}
\usepackage{hyperref}
\hypersetup{
    colorlinks=true,
    linkcolor=blue,
    filecolor=magenta,      
    urlcolor=blue,
}
\usepackage{textcomp}
%\usepackage[left=0.8in,right=1.0in,top=1.0in,bottom=1.0in]{geometry}
\usepackage{mathtools}
%\usepackage{minted}
\usepackage{gensymb}

\input{macros}


\begin{document}
\begin{align*}
f_i &= f_{\text{min}} + (f_{\text{max}} - f_{\text{min}}) \, \beta, \quad \beta \in [0, 1], \\
V_i^{t+1} &= V_i^t + (X_i^t + X^*) \, f_i, \\
X_i^{t+1} &= X_i^t + V_i^t,
\end{align*}

\textbf{Step 3.} If the random number is greater than \( r_i \), a new solution for the bat is generated by the following equation:

\begin{equation*}
X_{\text{new}} = X_{\text{old}} + \epsilon A^t,
\end{equation*}

where \( \epsilon \) is a random number, \( \epsilon \in [-1, 1] \), and \( A^t \) represents the average loudness of all bats at time \( t \).

\[
A_i^{t+1} = \alpha A_i^t,
\]
\[
r_i^t = r_i^0 \left[1 - e^{-\gamma t}\right],
\]

where \( A_i^{t+1} \) and \( A_i^t \) denote the loudness at times \( t \) and \( t+1 \), respectively; \( r_i^0 \) and \( r_i^t \) are the initial pulse rate and pulse rate at time \( t \), respectively, \( \alpha \) is a constant parameter in range \( [0, 1] \), \( \gamma \) is a constant parameter, and \( \gamma > 0 \). As \( t \rightarrow \infty \), \( A_i^t \rightarrow 0 \) and \( r_i^t \rightarrow r_i^0 \).

\[
\text{fitness} = \alpha \cdot \text{intraCluster} + \beta \cdot SC + \zeta \cdot \left( \frac{1}{PC} + CE \right)
\]

where:
\begin{itemize}
    \item \(\alpha\), \(\beta\), and \(\zeta\) are weighting coefficients.
    \item \(\text{intraCluster}\) represents the intra-cluster distance.
    \item \(SC\) is the Silhouette Coefficient, which measures the separation between clusters.
    \item \(PC\) is the Partition Coefficient, used to evaluate clustering fuzziness.
    \item \(CE\) is the Cluster Entropy, which measures the uncertainty in cluster assignments.
\end{itemize}


\end{document}
