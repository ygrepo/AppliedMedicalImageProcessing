\documentclass[12pt,twoside]{article}
\usepackage[dvipsnames]{xcolor}
\usepackage{tikz,graphicx,amsmath,amsfonts,amscd,amssymb,mathrsfs, bm,cite,epsfig,epsf,url}
\usepackage[hang,flushmargin]{footmisc}
\usepackage[colorlinks=true,urlcolor=blue,citecolor=blue]{hyperref}
\usepackage{amsthm,multirow,wasysym,appendix}
\usepackage{array,subcaption} 
% \usepackage[small,bf]{caption}
\usepackage{bbm}
\usepackage{pgfplots}
\usetikzlibrary{spy}
\usepgfplotslibrary{external}
\usepgfplotslibrary{fillbetween}
\usetikzlibrary{arrows,automata}
\usepackage{thmtools}
\usepackage{blkarray} 
\usepackage{textcomp}
\usepackage{float}
%\usepackage[left=0.8in,right=1.0in,top=1.0in,bottom=1.0in]{geometry}


\usepackage{times}
\usepackage{amsfonts}
\usepackage{amsmath}
\usepackage{latexsym}
\usepackage{color}
\usepackage{graphics}
\usepackage{enumerate}
\usepackage{amstext}
\usepackage{blkarray}
\usepackage{url}
\usepackage{epsfig}
\usepackage{bm}
\usepackage{hyperref}
\hypersetup{
    colorlinks=true,
    linkcolor=blue,
    filecolor=magenta,      
    urlcolor=blue,
}
\usepackage{textcomp}
%\usepackage[left=0.8in,right=1.0in,top=1.0in,bottom=1.0in]{geometry}
\usepackage{mathtools}
%\usepackage{minted}
\usepackage{gensymb}

\input{macros}


\begin{document}

\noindent Dr. Ardekani\\
EN.585.703.81.FA24 Applied Medical Image Processing\\
Module 7 Assignment\\
Johns Hopkins University\\
Student: Yves Greatti\\\


\section*{Question 1}
 A convenient form of 2D Radon transform is to use the following equation:
 \begin{equation}
    \{ \mathcal{R} \rho \}(t, \theta) = \int_{-\infty}^{+\infty} \int_{-\infty}^{+\infty} \rho(x, y) \delta(x \cos \theta + y \sin \theta - t) dx dy
  \end{equation}
  
  using this definition:
    \begin{enumerate}
        \item calculate $\{ \mathcal{R} \rho \}(t, 0)$ for $\rho(x, y) = \Pi\left(\frac{x}{a}\right)\Pi\left(\frac{y}{b}\right)$ (10 points), Where:
        \[
        \Pi(x) = 
        \begin{cases} 
        1, & \text{if } |x| < 1/2 \\ 
        0, & \text{otherwise}
        \end{cases}
        \]
        For $\theta=0$:
	\begin{equation}
            \{ \mathcal{R} \rho \}(t, 0) = \int_{-\infty}^{+\infty} \int_{-\infty}^{+\infty} \rho(x, y) \delta(x  - t) dx dy
	\end{equation}

	Since $ \int_{-\infty}^{+\infty}  \rho (x ,y) \delta(x  - t) dx =  \rho (x \; , y)$; then 
	 \begin{align*}
	 	 	\{ \mathcal{R} \rho \}(t, 0) &= \int_{-\infty}^{+\infty}  \int_{-\infty}^{+\infty} \Pi(\frac{x}{a}) \Pi(\frac{y}{b}) \; dx \; dy \\
			&=  \left( \int_{-\infty}^{+\infty}   \Pi(\frac{x}{a}) \; dx \right) \left( \int_{-\infty}^{+\infty}  \Pi(\frac{y}{b}) \; dy  \right)\\
			&=  \left( \int_{-\frac{a}{2}}^{\frac{a}{2}}   \Pi(\frac{x}{a}) \; dx \right) \left(\int_{-\frac{b}{2}}^{\frac{b}{2}}  \Pi(\frac{y}{b}) \; dy  \right)\\
			&= \left[ x\ \right]_{-\frac{a}{2}}^{\frac{a}{2}} \left[ x\ \right]_{-\frac{b}{2}}^{\frac{b}{2}} \\
			&= a \; \cdot  b
	\end{align*}

        \item Calculate and compare the Fourier transform of $\{ \mathcal{R} \rho \}(t, 45^{\circ})$ for a square object defined by:
        \[
        \rho(x, y) = \Pi\left(\frac{x}{2}\right)\Pi\left(\frac{y}{2}\right)
        \]
        using direct approach and projection-slice theorem (20 points).
        
        In both [a] and [b] show your work.
        
        Direct approach:  
     Given the function:

\[
\rho(x,y) = \Pi\left(\frac{x}{2}\right)\Pi\left(\frac{y}{2}\right)
\]

The 2D Fourier transform of $\rho(x,y)$ is defined as:

\[
\mathcal{F}\{\rho(x,y)\}(k_x, k_y) = \int_{-\infty}^{\infty}\int_{-\infty}^{\infty} \rho(x,y) e^{-i 2\pi(k_x x + k_y y)} dx\, dy
\]

Since $\rho(x,y)$ is non-zero only within the bounds $-1 \leq x \leq 1$ and $-1 \leq y \leq 1$, the integral reduces to:

\[
\mathcal{F}\{\rho(x,y)\}(k_x, k_y) = \int_{-1}^{1} \int_{-1}^{1} e^{-i 2\pi(k_x x + k_y y)} dx\, dy
\]

We can separate the integral into the product of two 1D Fourier transforms:

\[
\mathcal{F}\{\rho(x,y)\}(k_x, k_y) = \left(\int_{-1}^{1} e^{-i 2\pi k_x x} dx\right) \left(\int_{-1}^{1} e^{-i 2\pi k_y y} dy\right)
\]

Each of these integrals is a Fourier transform of the rectangular function:

\[
\int_{-1}^{1} e^{-i 2\pi k x} dx = \frac{\sin(2\pi k)}{\pi k} = 2 \cdot \text{sinc}(2k)
\]
with 
\[
 \text{sinc}(k) = \frac{\sin(\pi k)}{\pi k}
\]

To apply the Projection-Slice Theorem, we want to evaluate the 2D Fourier transform along the line \( k_x = k_y = \frac{k}{\sqrt{2}} \), corresponding to the \(45^\circ\) angle.


Substituting \( k_x = \frac{k}{\sqrt{2}} \) and \( k_y = \frac{k}{\sqrt{2}} \) into the Fourier transform expression:

\[
\mathcal{F}\{\rho(x, y)\}\left(\frac{k}{\sqrt{2}}, \frac{k}{\sqrt{2}}\right) = 4 \cdot \text{sinc}\left(2 \cdot \frac{k}{\sqrt{2}}\right) \cdot \text{sinc}\left(2 \cdot \frac{k}{\sqrt{2}}\right)
\]

Simplifying the arguments of the sinc functions:

The expression becomes:

\[
\mathcal{F}\{\rho(x, y)\}(k_x, k_y) = 4 \cdot \text{sinc}\left(\sqrt{2} k\right) \cdot \text{sinc}\left(\sqrt{2} k\right)
\]

Therefore, the The 2D Fourier transform of $\rho(x,y)$  at \(45^\circ\) angle is:

\[
\mathcal{F}\{\mathcal{R}\rho(t, 45^\circ)\}(k) = 4 \cdot  \text{sinc}^2\left(\sqrt{2} k\right) 
\]

We start first by calculating the Radon transform and then take its 1D Fourier transform:
\[
    \mathcal{F}\{ \mathcal{R} \rho \}(t, \theta) \}  =  \int_{-\infty}^{+\infty}  \left(  \int_{-\infty}^{+\infty} \int_{-\infty}^{+\infty} \rho(x, y) \delta(x \cos \theta + y \sin \theta - t) dx dy \right) e^{-i 2\pi k t} dt
\]

For \(\theta = 45^\circ\), and since \(\rho(x, y)\) is non-zero and equals 1 only when both \(|x| \leq 1\) and \(|y| \leq 1\), the equation becomes:

\[
\mathcal{F}\{\mathcal{R} \rho \}(t, 45^\circ) = \int_{-\infty}^{+\infty} \left( \int_{-1}^{1} \int_{-1}^{1} \delta(x \cos(45^\circ) + y \sin(45^\circ) - t) \, dx \, dy \right) e^{-i 2\pi k t} \, dt
\]

Swapping the order of integrals
\[
\mathcal{F}\{\mathcal{R} \rho \}(t, 45^\circ) = \int_{-1}^{1} \int_{-1}^{1} \left( \int_{-\infty}^{+\infty}  \delta(x \cos(45^\circ) + y \sin(45^\circ) - t)  e^{-i 2\pi k t} \, dt \right) \, dx \, dy 
\]


Substitute $t= \frac{x+y}{\sqrt{2}}$:
\begin{align*}
\mathcal{F}\{\mathcal{R} \rho \}(t, 45^\circ) &= \int_{-1}^{1} \int_{-1}^{1}   e^{-i 2\pi k \left( \frac{x+y}{\sqrt{2}} \right)}   \, dx \, dy \\
 &=  \left( \int_{-1}^{1}   e^{-i 2\pi \left( \frac{k}{\sqrt{2}} \right) x}   \, dx \right)  \left( \int_{-1}^{1}   e^{-i 2\pi \left( \frac{k}{\sqrt{2}} \right) y}   \, dy \right) \\
&=  \left( \int_{-1}^{1}   e^{-i 2\pi \left( \frac{k}{\sqrt{2}}  \right) \; x}   \, dx \right)^2
\end{align*}
For  $\Pi\left(\frac{x}{2}\right)$; its Fourier transform is
\begin{align*}
 \int_{-1}^{1}   e^{-i 2\pi \left( \frac{k}{\sqrt{2}}  \right) \; x}  \; dx  = 2  \text{sinc}(2 \frac{k}{\sqrt{2}})
  &= 2  \text{sinc}(\sqrt{2} k)
\end{align*}

Therefore
\[
\mathcal{F}\{\mathcal{R} \rho \}(t, 45^\circ) =  4 \cdot  \text{sinc}^2\left(\sqrt{2} k\right) 
\]


     \end{enumerate}
            
\section*{Question 2}

\noindent
Use Matlab to illustrate the projection-slice theorem by loading an MRI data set that is provided by Mathworks (\texttt{load mri}). 
We are working on slices 15 and 20 and angles 0 and 90 degrees (20 points).

\begin{enumerate}
    \item Perform the Radon transform on slice 15 using angles from 0 to 179 degrees (Matlab has a built-in function).
    \item Perform a 1D Fourier transform on the Radon-transformed signal from slice 15 for angles 0 and 90 degrees.
    \item Perform a 2D Fourier transform of slices 15 and 20.
    \item Compare the direct and projection-slice Fourier transforms for the two angles using slice 15. 
    For comparison, you can use the magnitude signals and plot them on top of each other.\\
    Answer: Looking at the two plots for angle 0 and 90 degrees; 
    the blue and red lines overlap almost perfectly in the two plots. The direct 2D Fourier transform and the 1D Fourier transform of the Radon-transformed signals
    are similar as stated by the projection-slice theorem.
    
    	
    \item Use the 1D Fourier transform on the Radon-transformed signal from slice 15 (for angles 0 and 90 degrees) and 
    compare it with the projection-slice Fourier transform of slice 20 (using the same angles and magnitude signal). 
    Plot the results and compare them with question 4.
    Make sure your code is fully documented and can be executed without error.\\
    Answer: The green and red lines do not quite overlap  compared to the previous question.
    They are the amplitude plots of  projection-slice Fourier transforms of different slices and we can expect differences in the amplitude of the signals.
   \end{enumerate}

\end{document}
