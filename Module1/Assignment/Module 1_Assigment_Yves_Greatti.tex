\documentclass[12pt,twoside]{article}
\usepackage[dvipsnames]{xcolor}
\usepackage{tikz,graphicx,amsmath,amsfonts,amscd,amssymb,mathrsfs, bm,cite,epsfig,epsf,url}
\usepackage[hang,flushmargin]{footmisc}
\usepackage[colorlinks=true,urlcolor=blue,citecolor=blue]{hyperref}
\usepackage{amsthm,multirow,wasysym,appendix}
\usepackage{array,subcaption} 
% \usepackage[small,bf]{caption}
\usepackage{bbm}
\usepackage{pgfplots}
\usetikzlibrary{spy}
\usepgfplotslibrary{external}
\usepgfplotslibrary{fillbetween}
\usetikzlibrary{arrows,automata}
\usepackage{thmtools}
\usepackage{blkarray} 
\usepackage{textcomp}
\usepackage{float}
%\usepackage[left=0.8in,right=1.0in,top=1.0in,bottom=1.0in]{geometry}


\usepackage{times}
\usepackage{amsfonts}
\usepackage{amsmath}
\usepackage{latexsym}
\usepackage{color}
\usepackage{graphics}
\usepackage{enumerate}
\usepackage{amstext}
\usepackage{blkarray}
\usepackage{url}
\usepackage{epsfig}
\usepackage{bm}
\usepackage{hyperref}
\hypersetup{
    colorlinks=true,
    linkcolor=blue,
    filecolor=magenta,      
    urlcolor=blue,
}
\usepackage{textcomp}
%\usepackage[left=0.8in,right=1.0in,top=1.0in,bottom=1.0in]{geometry}
\usepackage{mathtools}
%\usepackage{minted}
\usepackage{gensymb}

\input{macros}


\begin{document}

\noindent Dr. Ardekani\\
EN.585.703.81.FA24 Applied Medical Image Processing\\
Module 1 Assignment\\
Johns Hopkins University\\
Student: Yves Greatti\\\


\section*{Question 1 - Given the following measure values for the X-ray projection (see Fig. 1):}
\be
    \item Use the algebraic reconstruction technique to estimate values of \( u_1, \ldots, u_9 \) (5 points). Show your work for full credit.\\
    Since \( I_1 = u_1 + u_4 + u_7 = 12 \), and not knowing the distribution of \( u_1, u_4, \) and \( u_7 \), 
    we can assume that the coefficients are equal to 4. Likewise, for \( I_2 = u_2 + u_5 + u_8 = 15 \): \( u_2=u_5=u_8=5 \); and  \( I_3 = u_3+ u_6 + u_9 = 18 \): \( u_3=u_6=u_9=6 \). \\
  We now have:\\
  \[
\begin{array}{|c|c|c|}
\hline
4 & 5 & 6 \\
\hline
4 & 5 & 6 \\
\hline
4 & 5 & 6 \\
\hline
\end{array}
\]
However,  \( I_4 = u_1 + u_2+ u_3 = 6 \) but based on the current estimation, its value is 15 which is 9 more than what is supposed to be.  We take the difference and divide it by the number of elements in the row 
which is 3 and we make a correction to \( u_1, u_2 \) and \(u_3\) by subtracting 3 from each to get $u_1=1$, $u_2=2$ and $u_3=3$. 
$u_4 + u_5 + u_6 = 15$ which is the expected intensity $I_5$.  
$u_7+ u_8 + u_9 = 15$ which is 9 less than $I_6=24$. We divide the difference by 3, the number of elements in this row and add 3 to $u_4; u_5 ; u_6 $.
Their sum is $I_6$. The projection is:
 \[
\begin{array}{|c|c|c|}
\hline
1 & 2 & 3 \\
\hline
4 & 5 & 6 \\
\hline
7 & 8 & 9 \\
\hline
\end{array}
\]
Finally, $u_1+ u_5 + u_9 = 15$ which is $I_7$ but $u_3 + u_5 + u_7 = 15$ which is 1 more than $I_8=14$. 
We  divide this difference by 3, number of elements in the diagonal, and subtract $\frac{1}{3}$ to each element along this diagonal.

\[
\begin{array}{|c|c|c|}
\hline
1 & 2 &  \approx 2 .66\\
\hline
4 & \approx 4.66 & 6 \\
\hline
\approx 6.66 & 8 & 9 \\
\hline
\end{array}
\]
 \item What would you do if, after one iteration over all provided values, your sums are not equal to the measured values (2 points).
 
 After one iteration over all provided values, if the sums are not equal to the measured values; we will keep iterating over all the measured values
 using the same algorithm applied in question 1  and  stop the iterations if the values of the differences between the estimated sums and the measured values are below a given tolerance.
 If all the differences in absolute values are within a given tolerance, we will stop the iterations.
 
 \ee

 \section*{Question 2 -Given a line \( I_{t, \theta} : I_{\frac{1}{2}, \frac{\pi}{6}} \):}   
    \be
        \item Write its equation in standard form \( (x(s), y(s)) \) (5 points) (show your work).
        \begin{align*}
   		 I_{\frac{1}{2}, \frac{\pi}{6}} &= (x(s), y(s)) \\
    		&= \left\{ t \cos{\theta} - s \sin{\theta}, \; t \sin{\theta} + s \cos{\theta} \; \middle| \; t = \frac{1}{2}, \; \theta = \frac{\pi}{6}, \; s \in \mathbb{R} \right\} \\
	        &= \left\{ \frac{1}{2} \cos{\frac{\pi}{6}} - s \sin{\frac{\pi}{6}}, \; \frac{1}{2} \sin{\frac{\pi}{6}} + s \cos{\frac{\pi}{6}} \; \middle| \; s \in \mathbb{R} \right\} \\
    		&= \left\{ \frac{1}{2} \cdot \frac{\sqrt{3}}{2} - s \cdot \frac{1}{2}, \; \frac{1}{2} \cdot \frac{1}{2} + s \cdot \frac{\sqrt{3}}{2} \; \middle| \; s \in \mathbb{R} \right\} \\
    		&= \left\{ \frac{\sqrt{3}}{4} - \frac{s}{2}, \; \frac{1}{4} + s \cdot \frac{\sqrt{3}}{2} \; \middle| \; s \in \mathbb{R} \right\} \\
    		&= \left\{ \frac{1}{4} \left(\sqrt{3} - 2s\right), \; \frac{1}{4} \left(1 + 2\sqrt{3}s\right) \; \middle| \; s \in \mathbb{R} \right\}.
	\end{align*}
	
	\item Find the values of \( s \) where this line intersects the unit circle (5 points). Show your work. 

If \( P(x(s), y(s)) \) is on the line \( I_{\frac{1}{2}, \frac{\pi}{6}} \) and also on the unit circle, it must satisfy the equation of the circle:

\begin{align*}
    x(s)^2 + y(s)^2 &= 1, \\
    \left(\frac{1}{4}\left(\sqrt{3} - 2s\right)\right)^2 + \left(\frac{1}{4}\left(1 + 2\sqrt{3}s\right)\right)^2 &= 1, \\
    \left(\sqrt{3} - 2s\right)^2 + \left(1 + 2\sqrt{3}s\right)^2 &= 16, \\
    (3 - 4\sqrt{3}s + 4s^2) + (1 + 4\sqrt{3}s + 12s^2) &= 16, \\
    3 + 1 + 4s^2 + 12s^2 &= 16, \\
    16s^2 + 4 &= 16, \\
    16s^2 &= 12, \\
    s^2 &= \frac{12}{16}, \\
    s^2 &= \frac{3}{4}, \\
    s &= \pm \frac{\sqrt{3}}{2}.
\end{align*}

\ee


\end{document}
